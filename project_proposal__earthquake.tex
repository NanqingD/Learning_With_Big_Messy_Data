\documentclass[a4paper,12pt]{article}
\usepackage{amsmath,amssymb}
\begin{document}
\begin{center}
\Large{\textbf{ORIE 4741 Project Proposal}}\\
\vspace{10pt}
\end{center}
In earthquake engineering, response history analysis is widely used to analyze the performance of structures. It involves calculation of responses $X(t)$ of structural systems when the system is subjected to certain seismic inputs $A(t)$. We often use the maximum relative displacement of the response displacement as engineering demand parameter ($EDP$) to quantify structure performance. 

\begin{center}
          $A(t)$ $\Rightarrow$ \setlength{\fboxrule}{1.5pt}\fbox{\parbox[t]{4em}{Structure}} $\Rightarrow$ $X(t)$ $\Rightarrow$ $EDP = \max\limits_{0\leqslant t\leqslant \tau}|X(t)|$\\~\\
\end{center}

Structural systems can be perceived as complicated nonlinear systems with millions of degree of freedom, so it's computationally expensive and time-consuming to obtain the $EDP$ by solving differential equations of the structural systems. However, if we can find the hidden relationship between the input $A(t)$ and the $EDP$ without solving the differential equations, we can optimize the analysis process by significantly reducing the required computation time. 

In this project, we propose to find such relationship by using machine learning methods such as linear regression, nonparametric regression, support vector machine (SVM), and neuro network. The dataset we use contains seismic ground motions time series $\{A(t)\}$ simulated for a site in California, and the corresponding $\{EDP\}$ from a single-degree-of-freedom Bouc-Wen system. Now we have the input space $A(t)$ and the output $EDP$, we can find the relationship function $f$ given a dataset with $n=1000$ ground motions and corresponding $EDP$.

\end{document}